\documentclass[a4paper, 11pt]{report}

\usepackage[ left=2.5 cm, right=2.5 cm]{geometry}
\usepackage{csquotes}
\usepackage[german, ngerman]{babel}
\usepackage{graphicx}
\usepackage{float}
\usepackage{listings}
\usepackage{color}
\usepackage{amsmath}
\usepackage[hang]{footmisc}
\usepackage[onehalfspacing]{setspace}
\usepackage[utf8]{inputenc}
\usepackage{acronym}

\begin{document}
	\noindent
	\thispagestyle{empty}
	\Large\textbf{Praktikum DBMS - Stolperwege - iOS-App}\\
	\large\textbf{Anforderungsanalyse} \\ \newline
	\large \textbf{Team Mitglieder} \\ \newline
	Niclas G\"unther, s1304384\\
	Matrikelnummer: 5438994\\
	Email: niclas.guenther@t-online.de \\ \newline
	Matthias Kühnel, s4259738\\
	Matrikelnummer : 4792440 \\
	Email: s4259738@stud.uni-frankfurt.de \\
	\newpage
	
	\section*{Anforderungen}
	
	Im Rahmen des Praktikums soll eine Applikation für iOS-Geräte konzeptioniert und entwickelt werden, mit der es möglich ist, die Stolperwege innerhalb von Frankfurt zu erkunden, auf einer Karte aufzusuchen und mehr Informationen über die Personen zu erfahren, welchen die Steine zu Ehren des Holocaust gewidmet wurden.\\
	Da es bereits eine weit entwickelte Android-Applikation gibt und die iOS-App noch am Anfang ist, wird das primäre Ziel sein, diese auf den gleichen Stand zu bringen und optional zu erweitern, sofern der Fortschritt dies zulässt. In diesem Rahmen stellen wir ein mögliches Konzept vor, um die App umsetzen zu können. Dabei versuchen wir, uns möglichst an der Android-App zu orientieren, berücksichtigen jedoch, dass gewisse Inhalte (Listen, Karten, usw.) in iOS anders als in Android umgesetzt werden, so dass es trotzdem zu einigen Unterschieden kommen kann. Wir unterteilen einzelne Features in Menüpunkte, wodurch ein unabhängiges Arbeiten und Erstellen des Mockups der App möglich ist. Näheres dazu im Punkt "Menü".
	
	\section*{Konzeptvorstellung}
	
	Dieser Abschnitt stellt unseren Vorschlag vor, wie die App und deren Menüpunkte umgesetzt werden könnten. Für die Benutzung der App wird ein Login benötigt, die Möglichkeit zum Registrieren soll vorhanden sein. Inwiefern diese stattfinden wird und/oder nach welchen Formalia sie erfolgt überbleibt der Leitung.\\ Vorschlagen würden wir einen simplen Login mit Email-Adresse und Passwort, weitere Benutzerangaben können optional sein.
	
	\subsection*{Menü}
	
	Für einen besseren Überblick eignet sich ein Slide-Menü, bei welchem es möglich ist, verschiedene Features der App aufzurufen oder ggf. die Einstellungen zu ändern. Als Referenz kann man sich zum Vergleich das Menü von www.xing.com anzuschauen, welches dieses Menü exakt so wie wir uns dies vorstellen implementiert hat. Die Übersicht ist einfach, die Navigationspunkte werden klar definiert und dadurch haben wir auf dem Bildschirm mehr Platz für andere relevante Inhalte.
	
	\subsubsection*{Menüpunkt 1: Karten}
	
	Die einzelnen Stolpersteine sollen auf einer Karte angezeigt werden. Beim Klicken auf den einzelnen Stein ist es möglich, weitere Kurzinformationen über die Person anzeigen zu lassen, ein Tap auf den Infobutton soll noch mehr Inhalte darstellen (Wikipedia-Artikel, eine Art "Profil" mit den wichtigsten Daten oder Ähnliches, optional auch mit Weiterleitung zum Menüpunkt "Personen").
	
	\subsubsection{Menüpunkt 2 : Personen (optional)}
	
	Abhängig davon, wie gut die Umsetzung der App und wie die Datenbank ausgelegt ist besteht die Möglichkeit, Profile von Personen der Stolperwege anzeigen zu lassen. Diese können im Großen und Ganzen die grundlegenden Eckdaten anzeigen lassen, bei Bedarf bzw. unter Umständen weitere Informationen usw. darstellen.
	
	\subsubsection{Menüpunkt 2: Historische Karten}
	
	Da Frankfurt im zweiten Weltkrieg fast vollständig zerstört wurde und die heutige Stadtkarte der damaligen kaum ähnlich ist, soll versucht werden, diese Inhalte in einer passenden Form darstellen zu können. Inwiefern das Überlappen der alten Karte auf die Neue möglich ist, wird sich zeigen, da wir uns noch nicht so sehr mit Karten beschäftigt haben. 
	
	\subsubsection{Menüpunkt 3:  3D-Gebäude}
	
	Sofern möglich und unter iOS umsetzbar, werden unter diesem Menüpunkt historische Gebäude angezeigt, welche virtuell abgelaufen werden können. Näheres dazu vom Team welches sich mit \textit{Unity} beschäftigt.
	
	\subsubsection{Menüpunkt 4: Historische Wegbeschreibungen}
	Historische Wegbeschreibungen zeigen die Zeitlinie einer historischen Person. Hier soll es möglich sein, die eingetragenen Annotationen der einzelnen Personen chronlogisch geordnet anzusehen (Die Implementierung hierbei bzw. das Design wird noch vorgeschlagen).
	
	\subsubsection{Menüpunkt 5: Stammbäume}
	Abhängig von den vorhandenen Daten bzw. der Relationen könnte zu jeder Person der Stammbaum angezeigt werden. Die Tiefe dieses Baumes und die daraus resultierende Darstellung hängt von der Datenfülle bzw. dem Design-Konzept ab. 
	\subsubsection{Menüpunkt 6: Einstellungen}
	
	Hier sollen verschiedene Einstellungen möglich sein. Abhängig davon, was gewünscht ist, lassen sich verschiedene Komponenten einbauen. Das Passwort und der Nutzername sollen sich hier ändern lassen, weitere Informationen zum Profil können hier vorgenommen werden.
	
	\subsubsection{Schlusswort}
	
	Wie und auf welche Art die App umgesetzt werden kann wird warscheinlich sehr stark variieren, wie bereits mit Herrn Abrami besprochen könnte diese Absprache nach dem Ping-Pong-Prinzip ablaufen. Graphische Umsetzungen, die vielleicht schwer / nicht in der Zeit umzusetzen sind, müssten dann entweder ausgelassen (weil sie technisch nicht umsetzbar sind) oder auf das nächste Semester geschoben werden, wenn die nächste Gruppe an der App weiterarbeiten soll.
	
\end{document}